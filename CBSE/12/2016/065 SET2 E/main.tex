\documentclass[12pt,-letter paper]{article}                       
\usepackage{siunitx}                                              
\usepackage{setspace}
\usepackage{gensymb}                                              
\usepackage{xcolor}                                               
\usepackage{caption}
%\usepackage{subcaption}
\doublespacing                                                    
\singlespacing                                                    
\usepackage[none]{hyphenat}
\usepackage{amssymb}
\usepackage{relsize}
\usepackage[cmex10]{amsmath}
\usepackage{mathtools}
\usepackage{amsmath}                                              
\usepackage{commath}                                              
\usepackage{amsthm}
\interdisplaylinepenalty=2500
%\savesymbol{iint}\usepackage{txfonts}                                              
%\restoresymbol{TXF}{iint}                                        
\usepackage{wasysym}                                              
\usepackage{amsthm}
\usepackage{mathrsfs}                                             
\usepackage{txfonts}                                              
\let\vec\mathbf{}
\usepackage{stfloats}
\usepackage{float}
\usepackage{cite}
\usepackage{cases}                                                
\usepackage{subfig}                                               
%\usepackage{xtab}
\usepackage{longtable}
\usepackage{multirow}
%\usepackage{algorithm}
\usepackage{amssymb}
%\usepackage{algpseudocode}
\usepackage{enumitem}
\usepackage{mathtools}
%\usepackage{eenrc}
%\usepackage[framemethod=tikz]{mdframed}                          
\usepackage{listings}                                             
%\usepackage{listings}
\usepackage[latin1]{inputenc}
%%\usepackage{color}{
%%\usepackage{lscape}
\usepackage{textcomp}
\usepackage{titling}
\usepackage{hyperref}
%\usepackage{fulbigskip}
\usepackage{tikz}
\usepackage{graphicx}                                             
\lstset{
  frame=single,
  breaklines=true
}
\let\vec\mathbf{}
\usepackage{enumitem}                                             
\usepackage{graphicx}                                             
\usepackage{siunitx}
\let\vec\mathbf{}                                                 
\usepackage{enumitem}
\usepackage{graphicx}
\usepackage{enumitem}
\usepackage{tfrupee}
\usepackage{amsmath}
\usepackage{amssymb}
\usepackage{mwe} % for blindtext and example-image-a in example
\usepackage{esvect}
\usepackage{wrapfig}
\usepackage{physics}
\newcommand{\myvec}[1]{\ensuremath{\begin{pmatrix}#1\end{pmatrix}}}
\let\vec\mathbf

\newcommand{\mydet}[1]{\ensuremath{\begin{vmatrix}#1\end{vmatrix}}}
\providecommand{\brak}[1]{\ensuremath{\left(#1\right)}}
\providecommand{\cbrak}[1]{\ensuremath{\left\{#1\right\}}}
\providecommand{\sbrak}[1]{\ensuremath{\left[#1\right]}}

\begin{document}
%

\title{CBSE 12 2016 065 SET2 C}
\author{Devansh Jain - EE22BTECH11018}
	
	
%\title{
%	\logo{Matrix Analysis through Octave}{\begin{center}\includegraphics[scale=.24]{tlc}\end{center}}{}{HAMDSP}
%}


% paper title
% can use linebreaks \\ within to get better formatting as desired
%\title{Matrix Analysis through Octave}
%
%
% author names and IEEE memberships
% note positions of commas and nonbreaking spaces ( ~ ) LaTeX will not break
% a structure at a ~ so this keeps an author's name from being broken across
% two lines.
% use \thanks{} to gain access to the first footnote area
% a separate \thanks must be used for each paragraph as LaTeX2e's \thanks
% was not built to handle multiple paragraphs
%

%\author{<-this % stops a space
%\thanks{}}
%}
% note the % following the last \IEEEmembership and also \thanks - 
% these prevent an unwanted space from occurring between the last author name
% and the end of the author line. i.e., if you had this:
% 
% \author{....lastname \thanks{...} \thanks{...} }
%                     ^------------^------------^----Do not want these spaces!
%
% a space would be appended to the last name and could cause every name on that
% line to be shifted left slightly. This is one of those "LaTeX things". For
% instance, "\textbf{A} \textbf{B}" will typeset as "A B" not "AB". To get
% "AB" then you have to do: "\textbf{A}\textbf{B}"
% \thanks is no different in this regard, so shield the last } of each \thanks
% that ends a line with a % and do not let a space in before the next \thanks.
% Spaces after \IEEEmembership other than the last one are OK (and needed) as
% you are supposed to have spaces between the names. For what it is worth,
% this is a minor point as most people would not even notice if the said evil
% space somehow managed to creep in.



% The paper headers
%\markboth{Journal of \LaTeX\ Class Files,~Vol.~6, No.~1, January~2007}%
%{Shell \MakeLowercase{\textit{et al.}}: Bare Demo of IEEEtran.cls for Journals}
% The only time the second header will appear is for the odd numbered pages
% after the title page when using the twoside option.
% 
% *** Note that you probably will NOT want to include the author's ***
% *** name in the headers of peer review papers.                   ***
% You can use \ifCLASSOPTIONpeerreview for conditional compilation here if
% you desire.




% If you want to put a publisher's ID mark on the page you can do it like
% this:
%\IEEEpubid{0000--0000/00\$00.00~\copyright~2007 IEEE}
% Remember, if you use this you must call \IEEEpubidadjcol in the second
% column for its text to clear the IEEEpubid mark.



% make the title area
%\maketitle

%\tableofcontents



\renewcommand{\thefigure}{\theenumi}
\renewcommand{\thetable}{\theenumi}
%\renewcommand{\theequation}{\theenumi}

%\begin{abstract}
%%\boldmath
%In this letter, an algorithm for evaluating the exact analytical bit error rate  (BER)  for the piecewise linear (PL) combiner for  multiple relays is presented. Previous results were available only for upto three relays. The algorithm is unique in the sense that  the actual mathematical expressions, that are prohibitively large, need not be explicitly obtained. The diversity gain due to multiple relays is shown through plots of the analytical BER, well supported by simulations. 
%
%\end{abstract}
% IEEEtran.cls defaults to using nonbold math in the Abstract.
% This preserves the distinction between vectors and scalars. However,
% if the journal you are submitting to favors bold math in the abstract,
% then you can use LaTeX's standard command \boldmath at the very start
% of the abstract to achieve this. Many IEEE journals frown on math
% in the abstract anyway.

% Note that keywords are not normally used for peerreview papers.
%\begin{IEEEkeywords}
%Cooperative diversity, decode and forward, piecewise linear
%\end{IEEEkeywords}



% For peer review papers, you can put extra information on the cover
% page as needed:
% \ifCLASSOPTIONpeerreview
% \begin{center} \bfseries EDICS Category: 3-BBND \end{center}
% \fi
%
% For peerreview papers, this IEEEtran command inserts a page break and
% creates the second title. It will be ignored for other modes.
%\IEEEpeerreviewmaketitle


\begin{enumerate}
\section*{Algebra}
\item Solve for $x$ : $ \tan^{-1}\brak{\dfrac{2-x}{2+x}} = \dfrac{1}{2}\tan^{-1}\dfrac{x}{2}, x>0$.
\item Prove that $2\sin^{-1}\brak{\dfrac{3}{5}} - \tan^{-1}\brak{\dfrac{17}{31}} = \dfrac{\pi}{4}$.
\item Find the equation of the tangent line to the curve $y = \sqrt{5x - 3} - 5$, which is parallel to the line $4x - 2y + 5 = 0$.
\end{enumerate}

\begin{enumerate}
\section*{Vector Algebra}
\item If $\abs{\overrightarrow{a}} = 4, \abs{\overrightarrow{b}} = 3$ and $ \overrightarrow{a}.\overrightarrow{b} = 6\sqrt{3}$, then find the the value of $\abs{\overrightarrow{a}\times \overrightarrow{b}}$.
\item Write the coordinates of the point which is the reflection of the point $\brak{\alpha,\beta,\gamma}$ in the $XZ$-plane.
\item Find the position vector of the point which divides the join of points with position vectors $\overrightarrow{a} + 3\overrightarrow{b}$ and $\overrightarrow{a} - \overrightarrow{b}$ internally in the ratio $1:3$.
\item Show that the lines $\dfrac{x-1}{3} = \dfrac{y-1}{-1} = \dfrac{Z+1}{0}$ and $ \dfrac{x-4}{2} = \dfrac{y}{0} = \dfrac{z+1}{3}$ intersect. Find their point of intersection.
\item Find the angle between the vectors $ \overrightarrow{a} + \overrightarrow{b} $ and $ \overrightarrow{a} - \overrightarrow{b} $ if $ \overrightarrow = 2\hat{i} - \hat{j} +3\hat{k} $ and $ \overrightarrow{b} = 3\hat{i} + \hat{j} - 2\hat{k}$, and hence find a vector perpendicular to both $ \overrightarrow{a} + \overrightarrow{b} $ and $\overrightarrow{a} - \overrightarrow{b}$.
\item Find the coordinates of the foot of perpendicular and perpendicular distance from the point $P\brak{4,3,2}$ to the plane $x + 2y + 3z = 2$. Also find the image of $P$ in the plane.
\item Show that the relation $R$ defined by $\brak{a,b}$R$\brak{c,d} \Rightarrow a + d = b + c$ on the $A\times A$, where $A = \cbrak{1,2,3,\ldots,10}$ is an equivalence relation. Hence write the equivalence class $\sbrak{\brak{3,4}}; a,b,c,d \in A$.
\end{enumerate}

\begin{enumerate}
\section*{Matrices}
\item Write the number of all possible matrices of order $2 \times 3$ with each entry $1$ or $2$.
\item If $A = \myvec{1 & -2 & 3 \\ -4 & 2 & 5}$ and $B = \myvec{2 & 3 \\ 4 & 5 \\ 2 & 1}$ and $BA = \brak{b_{ij}}$, find $b_{21} + b{32}$.
\item Write the value of $\mydet{{a-b} & {b-c} & {c-a} \\ {b-c} & {c-a} & {a-b}\\ {c-a} & {a-b} & {b-c}}$.
\item On her birthday Seema decided to donate some money to children of an orphanage home. If there were $8$ children less, every one would have got \rupee$10$ more. However, if there were $16$ children more, every one would have got \rupee$10$ less. Using the matrix method, find the number of children and the amount distributed by Seema. What values are reflected by Seema's decision?
\item Solve for $x : \mydet{a+x & a-x & a-x \\ a-x & a+x & a-x \\ a-x & a-x & a+x} = 0$, using properties of determinants.
\item Using elementary row operations find the inverse of matrix $A = \myvec{3 & -3 & 4 \\ 2 & -3 & 4\\ 0 & -1 & 1}$ and hence solve the following system of equations $ 3x - 3y + 4z = 21$, $2x - 3y + 4z = 20$, $-y + z = 5$.
\end{enumerate}

\begin{enumerate}
\section*{Functions and Relations}
\item Show that the function f given by:
\begin{align*}
f\brak{x} &= \begin{cases}
	\dfrac{e^{\frac{1}{x}} - 1}{e^{\frac{1}{x}} + 1}, & x \neq 0\\
	-1 , & x = 0
	\end{cases}
\end{align*}
is discontinuous at $x=0$.
\item Verify Mean Value theorem for the function $f\brak{x} = 2\sin{x} + \sin{2x}$ on $\sbrak{0,\pi}$.
\item Find the intervals in which the function $f\brak{x} = \dfrac{4\sin x}{2 + \cos x} - x; 0 \leq x \leq 2\pi$ is strictly increasing or strictly decreasing.
\end{enumerate}

\begin{enumerate}
\section*{Integration}
\item Evaluate: $\int^5_1{\abs{x-1} + \abs{x-2} + \abs{x-3} dx}$
\item Evaluate: $ \int^{\pi}_0{\dfrac{x\sin{x}}{1 + 3\cos^2{x}}dx}$
\item Find: $\int{\brak{3x+5}\sqrt{5 + 4x - 2x^2}dx}$
\item Find: $\int{\dfrac{2x+1}{\brak{x^2 + 1}\brak{x^2 + 4}}dx}$
\item Using integration, find the area of the triangle formed by the negative $x$-axis and tangent and normal to the circle $x^2 + y^2 = 9$ at $\brak{-1,2\sqrt{2}}$.
\end{enumerate}

\begin{enumerate}
\section*{Differentiation}
\item Solve the differential equation:
\begin{align*}
	\brak{x^2 +3xy + y^2}dx - x^2dy &= 0
\end{align*}
given that $y=0$, when $x=1$.
\item If $x = e^{\cos{2t}}$ and $y = e^{\sin{2t}}$, prove that $ \dfrac{dy}{dx} = -\dfrac{y\log{x}}{x\log{y}}$.
\item Solve the differential equation:
\begin{align*}
x\dfrac{dy}{dx} + y - x + xy\cot{x} = 0; x \neq 0.
\end{align*}
\end{enumerate}

\begin{enumerate}
\section*{Probability}
\item A committee of $4$ students is selected at random from a group consisting of $7$ boys and $4$ girls. Find the probability that there are exactly $2$ boys in the committee, given that at least one girl must be there in the committee.
\item A random variable $X$ has the following probability distribution:
\begin{center}
\begin{tabular}{|c|c|c|c|c|c|c|c|}
\hline
$X$ & $0$ & $1$ & $2$ & $3$ & $4$ & $5$ & $6$\\
\hline
$P\brak{X}$ & $C$ & $2C$ & $2C$ & $3C$ & $C^2$ & $2C^2$ & $7C^2+C$\\
\hline
\end{tabular}
\end{center}
Find the value of $C$ and also calculate mean of the distribution.
\item $A$, $B$ and $C$ throw a pair of dice in that order alternately till one of them gets a total of $9$ and wins the game. Find their respective probabilities of winning, if $A$ starts first.
\end{enumerate}

\begin{enumerate}
\section*{Optimization}
\item A company manufactures two types of cardigans: type $A$ and type $B$. It costs \rupee$360$ to make a type $A$ cardigan and \rupee{120} to make a type $B$ cadigan. The company can make at most $300$ cardigans and spend at most \rupee$72000$ a day. The number of cardigans of type $B$ cannot exceed the number of cardigans of type $A$ by more than $200$. The company makes a profit of \rupee$100$ for each cardigan of type $A$ and \rupee$50$ for each cardigan of type $B$. Formulate this problem as a linear programming problem to maximize the profit to the company. Solve it graphically and find maximum profit.
\item Show that height of the cylinder of greatest volume which can be inscribed in a right circular cone of height $h$ and semi-vertical angle $\alpha$ is one third that of the cone and the greatest volume of the cyclinder is $\dfrac{4}{27}\pi h^3\tan^2\alpha$.
\end{enumerate}

\end{document}
