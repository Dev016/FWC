
\let\negmedspace\undefined
\let\negthickspace\undefined
\documentclass[journal,12pt,onecolumn]{IEEEtran}
\usepackage{cite}
\usepackage{amsmath,amssymb,amsfonts,amsthm}
\usepackage{algorithmic}
\usepackage{graphicx}
\usepackage{textcomp}
\usepackage{xcolor}
\usepackage{txfonts}
\usepackage{listings}
\usepackage{enumitem}
\usepackage{mathtools}
\usepackage{gensymb}
\usepackage[breaklinks=true]{hyperref}
\usepackage{tkz-euclide} % loads  TikZ and tkz-base
\usepackage{listings}
\usepackage{gvv}
\usepackage{tfrupee} 
%
%\usepackage{setspace}
%\usepackage{gensymb}
%\doublespacing
%\singlespacing

%\usepackage{graphicx}
%\usepackage{amssymb}
%\usepackage{relsize}
%\usepackage[cmex10]{amsmath}
%\usepackage{amsthm}
%\interdisplaylinepenalty=2500
%\savesymbol{iint}
%\usepackage{txfonts}
%\restoresymbol{TXF}{iint}
%\usepackage{wasysym}
%\usepackage{amsthm}
%\usepackage{iithtlc}
%\usepackage{mathrsfs}
%\usepackage{txfonts}
%\usepackage{stfloats}
%\usepackage{bm}
%\usepackage{cite}
%\usepackage{cases}
%\usepackage{subfig}
%\usepackage{xtab}
%\usepackage{longtable}
%\usepackage{multirow}
%\usepackage{algorithm}
%\usepackage{algpseudocode}
%\usepackage{enumitem}
%\usepackage{mathtools}
%\usepackage{tikz}
%\usepackage{circuitikz}
%\usepackage{verbatim}
%\usepackage{tfrupee}
%\usepackage{stmaryrd}
%\usetkzobj{all}
%    \usepackage{color}                                            %%
%    \usepackage{array}                                            %%
%    \usepackage{longtable}                                        %%
%    \usepackage{calc}                                             %%
%    \usepackage{multirow}                                         %%
%    \usepackage{hhline}                                           %%
%    \usepackage{ifthen}                                           %%
  %optionally (for landscape tables embedded in another document): %%
%    \usepackage{lscape}     
%\usepackage{multicol}
%\usepackage{chngcntr}
%\usepackage{enumerate}

%\usepackage{wasysym}
%\documentclass[conference]{IEEEtran}
%\IEEEoverridecommandlockouts
% The preceding line is only needed to identify funding in the first footnote. If that is unneeded, please comment it out.

\newtheorem{theorem}{Theorem}[section]
\newtheorem{problem}{Problem}
\newtheorem{proposition}{Proposition}[section]
\newtheorem{lemma}{Lemma}[section]
\newtheorem{corollary}[theorem]{Corollary}
\newtheorem{example}{Example}[section]
\newtheorem{definition}[problem]{Definition}
%\newtheorem{thm}{Theorem}[section] 
%\newtheorem{defn}[thm]{Definition}
%\newtheorem{algorithm}{Algorithm}[section]
%\newtheorem{cor}{Corollary}
\newcommand{\BEQA}{\begin{eqnarray}}
\newcommand{\EEQA}{\end{eqnarray}}
\newcommand{\define}{\stackrel{\triangle}{=}}
\theoremstyle{remark}
\newtheorem{rem}{Remark}

%\bibliographystyle{ieeetr}
\begin{document}
%

\bibliographystyle{IEEEtran}


\vspace{3cm}

\title{CBSE 12 2016 065 SET2 C}
\author{Devansh Jain - EE22BTECH11018}
	
	
%\title{
%	\logo{Matrix Analysis through Octave}{\begin{center}\includegraphics[scale=.24]{tlc}\end{center}}{}{HAMDSP}
%}


% paper title
% can use linebreaks \\ within to get better formatting as desired
%\title{Matrix Analysis through Octave}
%
%
% author names and IEEE memberships
% note positions of commas and nonbreaking spaces ( ~ ) LaTeX will not break
% a structure at a ~ so this keeps an author's name from being broken across
% two lines.
% use \thanks{} to gain access to the first footnote area
% a separate \thanks must be used for each paragraph as LaTeX2e's \thanks
% was not built to handle multiple paragraphs
%

%\author{<-this % stops a space
%\thanks{}}
%}
% note the % following the last \IEEEmembership and also \thanks - 
% these prevent an unwanted space from occurring between the last author name
% and the end of the author line. i.e., if you had this:
% 
% \author{....lastname \thanks{...} \thanks{...} }
%                     ^------------^------------^----Do not want these spaces!
%
% a space would be appended to the last name and could cause every name on that
% line to be shifted left slightly. This is one of those "LaTeX things". For
% instance, "\textbf{A} \textbf{B}" will typeset as "A B" not "AB". To get
% "AB" then you have to do: "\textbf{A}\textbf{B}"
% \thanks is no different in this regard, so shield the last } of each \thanks
% that ends a line with a % and do not let a space in before the next \thanks.
% Spaces after \IEEEmembership other than the last one are OK (and needed) as
% you are supposed to have spaces between the names. For what it is worth,
% this is a minor point as most people would not even notice if the said evil
% space somehow managed to creep in.



% The paper headers
%\markboth{Journal of \LaTeX\ Class Files,~Vol.~6, No.~1, January~2007}%
%{Shell \MakeLowercase{\textit{et al.}}: Bare Demo of IEEEtran.cls for Journals}
% The only time the second header will appear is for the odd numbered pages
% after the title page when using the twoside option.
% 
% *** Note that you probably will NOT want to include the author's ***
% *** name in the headers of peer review papers.                   ***
% You can use \ifCLASSOPTIONpeerreview for conditional compilation here if
% you desire.




% If you want to put a publisher's ID mark on the page you can do it like
% this:
%\IEEEpubid{0000--0000/00\$00.00~\copyright~2007 IEEE}
% Remember, if you use this you must call \IEEEpubidadjcol in the second
% column for its text to clear the IEEEpubid mark.



% make the title area
\maketitle

%\tableofcontents

\bigskip

\renewcommand{\thefigure}{\theenumi}
\renewcommand{\thetable}{\theenumi}
%\renewcommand{\theequation}{\theenumi}

%\begin{abstract}
%%\boldmath
%In this letter, an algorithm for evaluating the exact analytical bit error rate  (BER)  for the piecewise linear (PL) combiner for  multiple relays is presented. Previous results were available only for upto three relays. The algorithm is unique in the sense that  the actual mathematical expressions, that are prohibitively large, need not be explicitly obtained. The diversity gain due to multiple relays is shown through plots of the analytical BER, well supported by simulations. 
%
%\end{abstract}
% IEEEtran.cls defaults to using nonbold math in the Abstract.
% This preserves the distinction between vectors and scalars. However,
% if the journal you are submitting to favors bold math in the abstract,
% then you can use LaTeX's standard command \boldmath at the very start
% of the abstract to achieve this. Many IEEE journals frown on math
% in the abstract anyway.

% Note that keywords are not normally used for peerreview papers.
%\begin{IEEEkeywords}
%Cooperative diversity, decode and forward, piecewise linear
%\end{IEEEkeywords}



% For peer review papers, you can put extra information on the cover
% page as needed:
% \ifCLASSOPTIONpeerreview
% \begin{center} \bfseries EDICS Category: 3-BBND \end{center}
% \fi
%
% For peerreview papers, this IEEEtran command inserts a page break and
% creates the second title. It will be ignored for other modes.
%\IEEEpeerreviewmaketitle


\section{\textbf{A}}
\textbf{Question numbers 1 to 6 carry 1 mark each.}\\
\begin{enumerate}

	\item Write the number of vectors of unit length perpendicular to both the vectors $\vec{a} = \myvec{2 \\ 1 \\ 2}$ and $\vec{b} = \myvec{0\\ 1\\ 1}$.\\
	
	\item Write the number of all possible matrices of order $2\times2$ with each entry 1, 2 or 3.\\
	
	\item If $ x \in N$ and $\mydet{x+3 & -2 \\ -3x & 2x} = 8$, then find the value of $x$.\\
	
	\item Write the position vector of the point which divides the join of points with position vectors $3\vec{a} - 2\vec{b}$ and $2\vec{a} + 3\vec{b}$ in the ratio 2:1.\\
	
	\item Find the vector equation of the plane  with intercepts 3, -4 and 2 on $x, y$ and $z$ axis respectively.\\
	
	\item Use elementary column operation $ C_2 \rightarrow C_2  + 2C_1$ in the following matrix equation:\\
	$\myvec{2 & 1 \\ 2 & 0} = \myvec{3 & 1 \\ 2 & 0}\myvec{1 & 0 \\ -1 & 1}$\\
		
\section{\textbf{B}}
\textbf{Question numbers 7 to 19 carry 4 marks each.}\\

	\item The equation of tangent at $\myvec{2 \\ 3}$ on the curve $y^2 = ax^3 + b$ is $4x -5$. Find the values of $a$ and $b$. \\
	
	\item Find the co-ordinates of the point where the line through the points A $\myvec{3\\4\\1} $ and B $\myvec{5\\1\\6}$ crosses the XZ plane. Also find the angle which this line makes with the XZ plane.\\
	
	\item Find : $\int{\brak{3x+1}\sqrt{4 - 3x - 2x^2}dx}$\\
	
	\item The two adjacent sides of a parallelogram  are $\myvec{2\\-4\\-5}$ and $\myvec{2\\2\\3}$. Find the two unit vectors parallel to its diagnols. Using the diagnol vectors, find the area of the parallelogram.\\
	
	\item Form the differential equation of the family of circles in the second quadrant and touching the co-ordinate axes.\\
	
	\item In a game, a man wins \rupee 5 for getting a number greater than 4 and loses \rupee 1 otherwise, when a fair die is thrown. The man decided to throw a die thrice but to quit as when he gets a number greater than 4. Find the expected value of the amount he wins/loses.\\
	\textbf{OR}\\
	A bag contains 4 balls. Two balls are drawn at random (without replacement) and are found to be white. What is the probability that all balls in the bag are white?\\
	
	\item A trust invested some money in two types of bonds. The first bond pays $10\%$ interest and second bond pays $12\%$ interest. The trust received \rupee 2800 as interest. However if trust had interchanged money in bonds, they would have got \rupee 100 less as interest. Using matrix method, find the amount invested by the trust. Interest received on this amount will be given to Helpage India as donation. Which value is reflected in this question?\\
	
	\item Differentiate $x^{\sin{x}} + \brak{\sin{x}}^{\cos{x}}$ with respect to $x$.\\
	\textbf{OR}\\
	If $y = 2\cos{\brak{\log{x}}} + 3\sin{\brak{\log{x}}}$, prove that $x^2\frac{d^2y}{dx^2} + x\frac{dy}{dx} + y = 0$.\\
	
	\item Solve the equation for $x : \sin^{-1}{x} + \sin^{-1}{\brak{1-x}} = \cos^{-1}{x}$.\\
	\textbf{OR}\\
	If $ \cos^{-1}{\frac{x}{a}} + \cos^{-1}{\frac{y}{b}} = \alpha$, prove that $ \frac{x^2}{a^2} - 2\frac{xy}{ab}\cos{\alpha} + \frac{y^2}{b^2} = \sin^2{\alpha}$.\\
	
	\item If $x = a\sin{2t}\brak{1 + \cos{2t}}$ and $y = b\cos{2t}\brak{1 - \cos{2t}}$, find $ \frac{dy}{dx}$ at $t = \frac{\pi}{4}$.\\
	
	\item Solve the differential equation: $y + x\frac{dy}{dx} = x - y\frac{dy}{dx}$\\
	
	\item Evaluate: $ \int^{\frac{\pi}{2}}_0{\frac{\sin^2{x}}{\sin{x} + \cos{x}}dx}$\\
	\textbf{OR}\\
	Evaluate: $\int^{\frac{3}{2}}_0{|x\cos{\pi x}|dx}$\\
	
	\item Find: $\int{\frac{x^2}{x^4 + x^2 -2}}$\\
	
\section{\textbf{C}}
\textbf{Question numbers 20 to 26 carry 6 marks each.}\\

	\item Using properties of determinants, show that $\triangle{ABC}$ is isosceles if:\\
	$\mydet{1 & 1 & 1 \\ 1 + \cos{A} & 1 + \cos{B} & 1 + \cos{C} \\ \cos^2{A} + \cos{A} & \cos^2{B} + \cos{B} & \cos^2{C} + \cos{C} } = 0$\\
	\textbf{OR}\\
	A shopkeeper has 3 varieties of pens `A', `B' and `C'. Meenu purchased 1 pen of each variety for a total of \rupee{21}. Jeevan purchased 4 pens of `A' variety, 3 pens of `B' variety and 2 pens of `C' variety for \rupee{60}. While Shikha purchased 6 pens of `A' variety, 2 pens of `B' variety and 3 pens of `C' variety for \rupee{70}. Using matrix method, find cost of each variety of pen.\\
	
	\item There are two types of fertilizers `A' and `B'. `A' consists of $12\%$ nitrogen and $5\%$ phosphoric acid whereas `B' consists of $4\%$ nitrogen and $5\%$ phosphoric acid. After testing the soil conditions, farmer finds that he needs at least 12 kg of nitrogen and 12 kg of phosphoric acid for his crops. If `A' costs \rupee{10} per kg and `B' cost \rupee{8} per kg, then graphically determine how much of each type of fertiliser should be used so that nutrient requirements are met at a minimum cost.\\
	
	\item Prove that the least perimeter of an isoceles triangle in which a circle of radius r can be inscribed is $6\sqrt{3}r$.\\
	\textbf{OR}\\
	If the sum of lengths of hypotenuse and a side of a right angled triangle is given, show that area of triangle is maximum, when the angle between them is $\frac{\pi}{3}$.\\
	
	\item Five bad oranges are accidentally mixed with 20 good ones. If four oranges are drawn one by one successively with replacement, then find the probability distribution of number of bad oranges drawn. Hence find the mean and variance of the distribution.\\
	
	\item Prove that the curves $y^2 = 4x$ and $x^2 = 4y$ divide the area of square bounded by $x=0, x=4, y=4$ and $y=0$ into three equal parts.\\
	
	\item Show that the binary operation *  on $A = \textbf{R} - \cbrak{-1}$ defined as $a*b = a + b + ab$ for all $a, b \in A$  is commutative and associative on A. Also find the identity element of * in $A$ and prove that every element of $A$ is invertible.\\
	
	\item Find the position vector of the foot of perpendicular and the perpendicular distance from the point $P$ with position vector $\myvec{2\\3\\4}$ to the plane $\vec{r}.\myvec{2\\3\\4} - 26 =0$. Also find the image of $P$ in the plane.
	
	
	
	
\end{enumerate}


\end{document}


